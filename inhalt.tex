%% In dieses File kommt der Eigentliche Inhalt des Dokuments.
% \vspace{1cm}
%\begin{tikzpicture}
%    \draw[step=.5cm,black,line width=0.2mm,line cap=round,dash pattern=on
%    0.05mm off 4.95mm] (-.5,0) grid (14,4);
%\end{tikzpicture}
\section{Aufgabe}
\textit{Gewichtung der Aufgabe 39\% \textcolor{red}{7 Punkte}}

Zur Finanzierung eines Investitionsprojektes stehen insgesamt maximal
CHF~1\,000\,000.- zur Verfügung. Die Unternehmung rechnet grundsätzlich
mit einem kalkulatorischen Zins von 10\%.

Variante A stellt sich folgendermassen dar:

\begin{tabular}[h]{lr}
  Anschaffungspreis&400\,000\\
  Gewinn& 20\,000\\
\end{tabular}

Variante B basiert auf folgenden Grundlagen:

\begin{tabular}[H]{lr}
  Anschaffungspreis&300\,000\\
  Liquidationserlös&50\,000\\
  Nutzungsdauer&5 Jahre\\
  Laufende Betriebskosten&8\,000\\
  Erlös&93\,500\\
\end{tabular}

Geben Sie zu Handen der Geschäftsleitung eine begründete Empfehlung ab.
Legen Sie dazu allfällige Berechnungen transparent dar.

\color{red}


\begin{tabular}[H]{lr}
  \multicolumn{2}{l}{\textbf{Variante A}}\\
  Anschaffungspreis&400\,000\\
  Gewinn& 20\,000\\
  Rendite&10\%\\
\end{tabular}

\vspace{3mm}

\begin{tabular}[H]{lr}
  \multicolumn{2}{l}{\textbf{Variante B}}\\
  Anschaffungspreis&300\,000\\
  Liquidationserlös&50\,000\\
  Nutzungsdauer&5 Jahre\\
  Laufende Betriebskosten&8\,000\\
  mittlerer Zins&17\,500\\
  Abschreibungen&50\,000\\
  Erlös&93\,500\\
  Gewinn&18\,000\\
  Rendite&10.30\%\\
\end{tabular}

Variante B ist vorzuziehen. Sie weist eine höhere Rendite auf und könnte
mit dem zur Verfügung stehenden Kapital drei Mal umgesetzt werden.

Noch besser wäre eine Kombination von zwei Maschinen der Variante B mit
einer Maschine der Variante A.

\itshape
Korrekturanweisung: Darstellung 1 Punkt, mittlerer Zins 1 Punkt,
Abschreibungen 1 Punkt, Gewinn 1 Punkt, Rendite 1 Punkt, traditionelle
Empfehlung 1 Punkt, Kombination 2 Punkte
\normalfont

\color{black}

\section{Aufgabe}
\textit{Gewichtung der Aufgabe 33\% \textcolor{red}{6 Punkte}}

Vergleichen Sie zwei Projekte, von denen folgende Informationen
vorliegen:

\vspace{3mm}

\begin{tabular}[H]{lrlr}
  \multicolumn{2}{l}{\textbf{Variante A:}}&\multicolumn{2}{l}{\textbf{Variante B:}}\\
  Investiertes Kapital:&100\,000&          Investiertes Kapital:&130\,000\\
  jährliche Rückflüsse:&50\,000&           jährliche Rückflüsse:&39\,000\\
  Nutzungsdauer:&3 Jahre&                  Nutzungsdauer:&6 Jahre\\
  kalkulatorischer Zins:&7.5\%&            kalkulatorischer Zins:&7.5\%\\
\end{tabular}

\vspace{3mm}

Zeigen Sie mit Hilfe transparenter Berechnungen, welches Projekt Sie der
Geschäftsleitung empfehlen.

\color{red}
\begin{adjustwidth}{-2cm}{}
\begin{tabular}[H]{lrrrrrrrrr}
  \toprule
  $i=.075$&$K_0$&$NPV$&$PV$&1&2&3&4&5&6\\
  $v$&&&&$\frac{1}{1.075^1}$&$\frac{1}{1.075^2}$&$\frac{1}{1.075^3}$&$\frac{1}{1.075^4}$&$\frac{1}{1.075^5}$&$\frac{1}{1.075^6}$\\
  &&&&0.9302&0.8653&0.8050&0.7488&0.6966&0.6480\\
  \midrule
  Variante
  A&100\,000&30\,026&130\,026&50\,000&50\,000&50\,000&---&---&---\\
  \multicolumn{4}{l}{CF Variante A abgezinst}&46512&43267&40248&---&---&---\\
  \multicolumn{2}{l}{NPV A kumuliert}&54\,196&&&&&&&\\
  \midrule
  Variante
  B&130\,000&53\,060&183\,060&39\,000&39\,000&39\,000&39\,000&39\,000&39\,000\\
  \multicolumn{4}{l}{CF Variante A
  abgezinst}&36279&33748&31393&29203&27166&25270\\
 \bottomrule 
\end{tabular}
\end{adjustwidth}

Variante A hat zwar bei nur einem Durchlauf den tieferen Net Present
Value, da das Projekt aber zweimal hintereinander laufen kann wird der
kumulierte NPV grösser als jener der Variante B.

\itshape
Korrekturanweisung: Darstellung 1 Punkt, Berechnung A 1 Punkt, Berechnung A Phase 2 1 Punkt,
Berechnung B 1 Punkt, Empfehlung basierend auf einem Durchlauf 1 Punkt,
Empfehlung basierend auf den kumulierten NPV 2 Punkte.
\normalfont
\color{black}

\section{Aufgabe}
\textit{Gewichtung der Aufgabe 28\% \textcolor{red}{5 Punkte}}

Unterziehen Sie das Investitionsrechnungsverfahren der Internal Rate of
Return (IRR) einer kritischen Würdigung.

\color{red}
Mit der Internal Rate of Return wird berechnet zu welchem
kalkulatorischen Zins der Net Present Value gerade Null ist (1 Pt).
Eine Stärke des IRR liegt darin, dass Projekte mit unterschiedlichen
Volumina und Laufzeiten miteinander vergleichbar werden (1 Pt).
Eine Schwäche liegt allerdings darin, dass Grössen die ohnehin nur
geschätzt sind auf mehrere Stellen hinter dem Koma gerechnet werden
können und so eine Scheingenauigkeit ergeben (1 Pt).
Insgesamt ist das Verfahren gemessen am Erkenntnisgewinn wohl zu
aufwändig (1 Pt).

\itshape
Korrekturanweisung: Sprache 1 Punkt, Beschreibung des Phänomens 1 Punkt, eine plausible
Stärke 1 Punkt, eine plausible Schwäche 1 Punkt, konsequentes Fazit 1
Punkt
\normalfont

\includepdf[landscape=true]{abzinsungstabelle.pdf}
